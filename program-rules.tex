\documentclass{ross}

\usepackage{enumerate}
\usepackage{enumitem}

\title{Program Rules}
\begin{document}
\maketitle
Participants are {\bf not} allowed to bring the following items to the Ross
Program:
\begin{itemize}
\item Televisions, DVD players, music players with speakers, or similar systems,
\item Computers, iPads, tablets, or electronic games,
\item Valuable items or large amounts of cash (there are ATMs on campus),
\item Automobiles, bicycles, skateboards, etc.
\end{itemize}



\bigskip\hrule
\textbf{Cell Phones:} Students may bring cell phones, but there are restrictions. 
\begin{enumerate}[label=(\roman*),itemsep=0.5em,topsep= 0em]

\item {\it Classes.}  All phones must be off during classes or lectures. 

\item {\it Telephone and text.}  Do not make phone calls or send text messages 
where those actions might distract other people.

\item {\it Internet.} Avoid internet use during the Ross Program!  It is tempting to read newsfeeds, play online games, watch videos, read (or write) blogs, find friends on Facebook, and surf through websites.  But such activities are huge time-wasters, and will distract you and your friends from concentrating on the challenging math problems we thrive on.  \\[5pt]
Working with math-oriented websites can {\it seem} productive. But if you're working on a hard Ross problem and then read related ideas online or in a book, you are under-cutting your own process of discovery. Work through the problems by your own efforts, and by having mathematial conversations with colleagues, JCs, and counselors.  Ross's central goal  is to help you learn how to think like a mathematician.\\
\hspace*{1cm} Solutions are not as satisfying if you looked up the answers.

\item {\it Camera.}  It is fine to take photos of friends, and enables you to record and remember fun times.  HOWEVER, some photos are not appropriate.  Be sure that anyone in a photo you take has given you permission for that picture.  This is especially important for any photos you send to others, or images that you will post on social media.  

\item{\it  GPS.}  When you arrive in Columbus, a Ross staff member will help you to install a tracking program on your phone.  Then the Ross administrators can find you in emergencies.  Of course you may remove that app at the end of the six weeks.
\end{enumerate}

\bigskip\hrule
\pagebreak

To help ensure basic safety, students should be supervised 
by Ross Counselors at all times.
Ross participants should not leave the dormitory after dark unless going to a
scheduled Ross activity, and in the company of a counselor.

DO NOT LEAVE THE CAMPUS of the Ohio Dominican University unless \\
absolutely necessary, and in the company of a Counselor or other responsible adult.  

\underline{Off-campus Excursions}.  To leave campus with
an adult friend or family member, you must complete a {\it Leave Request Form} 
and have it approved by the Head Counselor prior to departure.
We work hard to build an environment where students can focus on hard math problems
without distractions, maintain their mathematical momentum,
 and build a strong feeling of community within the dormitory.  
 A trip away from the Ross community will decrease that momentum.

\bigskip

Standard rules regarding respect and behavior apply in the Ross
Program, just as they do in high schools. These basic rules of civil society hold 
for the duration of our summer program, both in and out of the classroom. 
Please review the Ross Program's \textit{Standards of Behavior} for
a clearer statement of what behaviors this involves.
\end{document}

%%% Local Variables:
%%% mode: latex
%%% TeX-master: t
%%% End:
